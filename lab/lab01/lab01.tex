% Options for packages loaded elsewhere
\PassOptionsToPackage{unicode}{hyperref}
\PassOptionsToPackage{hyphens}{url}
\documentclass[
]{article}
\usepackage{xcolor}
\usepackage[margin=1in]{geometry}
\usepackage{amsmath,amssymb}
\setcounter{secnumdepth}{-\maxdimen} % remove section numbering
\usepackage{iftex}
\ifPDFTeX
  \usepackage[T1]{fontenc}
  \usepackage[utf8]{inputenc}
  \usepackage{textcomp} % provide euro and other symbols
\else % if luatex or xetex
  \usepackage{unicode-math} % this also loads fontspec
  \defaultfontfeatures{Scale=MatchLowercase}
  \defaultfontfeatures[\rmfamily]{Ligatures=TeX,Scale=1}
\fi
\usepackage{lmodern}
\ifPDFTeX\else
  % xetex/luatex font selection
\fi
% Use upquote if available, for straight quotes in verbatim environments
\IfFileExists{upquote.sty}{\usepackage{upquote}}{}
\IfFileExists{microtype.sty}{% use microtype if available
  \usepackage[]{microtype}
  \UseMicrotypeSet[protrusion]{basicmath} % disable protrusion for tt fonts
}{}
\makeatletter
\@ifundefined{KOMAClassName}{% if non-KOMA class
  \IfFileExists{parskip.sty}{%
    \usepackage{parskip}
  }{% else
    \setlength{\parindent}{0pt}
    \setlength{\parskip}{6pt plus 2pt minus 1pt}}
}{% if KOMA class
  \KOMAoptions{parskip=half}}
\makeatother
\usepackage{color}
\usepackage{fancyvrb}
\newcommand{\VerbBar}{|}
\newcommand{\VERB}{\Verb[commandchars=\\\{\}]}
\DefineVerbatimEnvironment{Highlighting}{Verbatim}{commandchars=\\\{\}}
% Add ',fontsize=\small' for more characters per line
\usepackage{framed}
\definecolor{shadecolor}{RGB}{248,248,248}
\newenvironment{Shaded}{\begin{snugshade}}{\end{snugshade}}
\newcommand{\AlertTok}[1]{\textcolor[rgb]{0.94,0.16,0.16}{#1}}
\newcommand{\AnnotationTok}[1]{\textcolor[rgb]{0.56,0.35,0.01}{\textbf{\textit{#1}}}}
\newcommand{\AttributeTok}[1]{\textcolor[rgb]{0.13,0.29,0.53}{#1}}
\newcommand{\BaseNTok}[1]{\textcolor[rgb]{0.00,0.00,0.81}{#1}}
\newcommand{\BuiltInTok}[1]{#1}
\newcommand{\CharTok}[1]{\textcolor[rgb]{0.31,0.60,0.02}{#1}}
\newcommand{\CommentTok}[1]{\textcolor[rgb]{0.56,0.35,0.01}{\textit{#1}}}
\newcommand{\CommentVarTok}[1]{\textcolor[rgb]{0.56,0.35,0.01}{\textbf{\textit{#1}}}}
\newcommand{\ConstantTok}[1]{\textcolor[rgb]{0.56,0.35,0.01}{#1}}
\newcommand{\ControlFlowTok}[1]{\textcolor[rgb]{0.13,0.29,0.53}{\textbf{#1}}}
\newcommand{\DataTypeTok}[1]{\textcolor[rgb]{0.13,0.29,0.53}{#1}}
\newcommand{\DecValTok}[1]{\textcolor[rgb]{0.00,0.00,0.81}{#1}}
\newcommand{\DocumentationTok}[1]{\textcolor[rgb]{0.56,0.35,0.01}{\textbf{\textit{#1}}}}
\newcommand{\ErrorTok}[1]{\textcolor[rgb]{0.64,0.00,0.00}{\textbf{#1}}}
\newcommand{\ExtensionTok}[1]{#1}
\newcommand{\FloatTok}[1]{\textcolor[rgb]{0.00,0.00,0.81}{#1}}
\newcommand{\FunctionTok}[1]{\textcolor[rgb]{0.13,0.29,0.53}{\textbf{#1}}}
\newcommand{\ImportTok}[1]{#1}
\newcommand{\InformationTok}[1]{\textcolor[rgb]{0.56,0.35,0.01}{\textbf{\textit{#1}}}}
\newcommand{\KeywordTok}[1]{\textcolor[rgb]{0.13,0.29,0.53}{\textbf{#1}}}
\newcommand{\NormalTok}[1]{#1}
\newcommand{\OperatorTok}[1]{\textcolor[rgb]{0.81,0.36,0.00}{\textbf{#1}}}
\newcommand{\OtherTok}[1]{\textcolor[rgb]{0.56,0.35,0.01}{#1}}
\newcommand{\PreprocessorTok}[1]{\textcolor[rgb]{0.56,0.35,0.01}{\textit{#1}}}
\newcommand{\RegionMarkerTok}[1]{#1}
\newcommand{\SpecialCharTok}[1]{\textcolor[rgb]{0.81,0.36,0.00}{\textbf{#1}}}
\newcommand{\SpecialStringTok}[1]{\textcolor[rgb]{0.31,0.60,0.02}{#1}}
\newcommand{\StringTok}[1]{\textcolor[rgb]{0.31,0.60,0.02}{#1}}
\newcommand{\VariableTok}[1]{\textcolor[rgb]{0.00,0.00,0.00}{#1}}
\newcommand{\VerbatimStringTok}[1]{\textcolor[rgb]{0.31,0.60,0.02}{#1}}
\newcommand{\WarningTok}[1]{\textcolor[rgb]{0.56,0.35,0.01}{\textbf{\textit{#1}}}}
\usepackage{graphicx}
\makeatletter
\newsavebox\pandoc@box
\newcommand*\pandocbounded[1]{% scales image to fit in text height/width
  \sbox\pandoc@box{#1}%
  \Gscale@div\@tempa{\textheight}{\dimexpr\ht\pandoc@box+\dp\pandoc@box\relax}%
  \Gscale@div\@tempb{\linewidth}{\wd\pandoc@box}%
  \ifdim\@tempb\p@<\@tempa\p@\let\@tempa\@tempb\fi% select the smaller of both
  \ifdim\@tempa\p@<\p@\scalebox{\@tempa}{\usebox\pandoc@box}%
  \else\usebox{\pandoc@box}%
  \fi%
}
% Set default figure placement to htbp
\def\fps@figure{htbp}
\makeatother
\setlength{\emergencystretch}{3em} % prevent overfull lines
\providecommand{\tightlist}{%
  \setlength{\itemsep}{0pt}\setlength{\parskip}{0pt}}
\usepackage{bookmark}
\IfFileExists{xurl.sty}{\usepackage{xurl}}{} % add URL line breaks if available
\urlstyle{same}
\hypersetup{
  pdftitle={Lab 01: Welcome to Data Science with R},
  hidelinks,
  pdfcreator={LaTeX via pandoc}}

\title{Lab 01: Welcome to Data Science with R}
\author{}
\date{\vspace{-2.5em}}

\begin{document}
\maketitle

\section{Welcome!}\label{welcome}

Welcome to DATA1517 Each week you will complete a lab assignment like
this one. You can't learn technical subjects without hands-on practice,
so labs are an important part of the course.

Before we get started, there are some administrative details.

The weekly lab sessions are designed to develop your skills with
computational and inferential concepts. These lab assignments are a
required part of the course and will be released every Monday and due
next Monday at 5 PM.

Lab sessions are not webcast! Here are the policies for getting full
credit (Lab is worth 10\% of your final grade):

\begin{enumerate}
\def\labelenumi{\arabic{enumi}.}
\item
  If you are within regular lab, to receive credit for lab you must
  attend.
\item
  If you are within the self-service lab you will have the option to
  complete the lab on your own and submit the completed lab by 5 PM on
  the due date.
\end{enumerate}

Collaborating on labs is more than okay -- it's encouraged! You should
rarely remain stuck for more than a few minutes on questions in labs, so
ask an instructor or classmate for help. (Explaining things is
beneficial, too -- the best way to solidify your knowledge of a subject
is to explain it.) Please don't just share answers, though.

\paragraph{Today's lab}\label{todays-lab}

In today's lab, you'll learn how to:

\begin{enumerate}
\def\labelenumi{\arabic{enumi}.}
\tightlist
\item
  navigate R Markdown documents (like this one);
\item
  write and evaluate some basic \emph{expressions} in R, the statistical
  computing language;
\item
  call \emph{functions} to use code other people have written; and
\item
  break down R code into smaller parts to understand it.
\end{enumerate}

This lab covers parts of
\href{http://www.inferentialthinking.com/chapters/03/programming-in-python.html}{Chapter
3} of the online textbook, adapted for R. You should read the examples
in the book, but not right now. Instead, let's get started!

\section{1. R Markdown Documents}\label{r-markdown-documents}

This document is called an R Markdown document. A document is a place to
write programs and view their results, and also to write text.

\subsection{1.1. Text sections}\label{text-sections}

In an R Markdown document, each section containing text or code is
organized into chunks. Text sections (like this one) can be edited by
modifying the source. They're written in a simple format called
\href{https://www.markdownguide.org/basic-syntax/}{Markdown} to add
formatting and section headings. You don't need to learn Markdown, but
you might want to.

After you edit text, you can knit the document to see the changes. (Try
not to delete the instructions of the lab.)

\textbf{Question 1.1.1.} This paragraph is in its own text section. Try
editing it so that this sentence is the last sentence in the paragraph,
and then knit the document. This sentence, for example, should be
deleted. So should this one.

\subsection{1.2. Code chunks}\label{code-chunks}

Other sections contain code in the R language. Running a code chunk will
execute all of the code it contains.

To run the code in a code chunk, click the green ``Run Chunk'' button in
the top right of the chunk, or place your cursor in the chunk and press
Ctrl+Shift+Enter (Cmd+Shift+Enter on Mac).

Try running this chunk:

\begin{Shaded}
\begin{Highlighting}[]
\FunctionTok{print}\NormalTok{(}\StringTok{"Hello, World!"}\NormalTok{)}
\end{Highlighting}
\end{Shaded}

\begin{verbatim}
## [1] "Hello, World!"
\end{verbatim}

And this one:

\begin{Shaded}
\begin{Highlighting}[]
\FunctionTok{print}\NormalTok{(}\StringTok{"👋, 🌏!"}\NormalTok{)}
\end{Highlighting}
\end{Shaded}

\begin{verbatim}
## [1] "👋, 🌏!"
\end{verbatim}

The fundamental building block of R code is an expression. Chunks can
contain multiple lines with multiple expressions. When you run a chunk,
the lines of code are executed in the order in which they appear. Every
\texttt{print} expression prints a line. Run the next chunk and notice
the order of the output.

\begin{Shaded}
\begin{Highlighting}[]
\FunctionTok{print}\NormalTok{(}\StringTok{"First this line is printed,"}\NormalTok{)}
\end{Highlighting}
\end{Shaded}

\begin{verbatim}
## [1] "First this line is printed,"
\end{verbatim}

\begin{Shaded}
\begin{Highlighting}[]
\FunctionTok{print}\NormalTok{(}\StringTok{"and then this one."}\NormalTok{)}
\end{Highlighting}
\end{Shaded}

\begin{verbatim}
## [1] "and then this one."
\end{verbatim}

\textbf{Question 1.2.1.} Change the chunk above so that it prints out:

\begin{verbatim}
First this line,
then the whole 🌏,
and then this one.
\end{verbatim}

\emph{Hint:} If you're stuck on the Earth symbol for more than a few
minutes, try talking to a neighbor or a staff member. That's a good idea
for any lab problem.

\subsection{1.3. Writing R Markdown
documents}\label{writing-r-markdown-documents}

You can use R Markdown documents for your own projects or documents.
When you make your own document, you'll need to create your own chunks
for text and code.

To add a code chunk, you can use the Insert menu or press Ctrl+Alt+I
(Cmd+Option+I on Mac), or manually add it by typing \texttt{\{r\}\ and}.

\textbf{Question 1.3.1.} Add a code chunk below this section. Write code
in it that prints out:

\begin{verbatim}
A whole new chunk! ♪🌏♪
\end{verbatim}

(That musical note symbol is like the Earth symbol. You can copy it from
here: ♪)

Run your chunk to verify that it works.

\subsection{1.4. Errors}\label{errors}

R is a language, and like natural human languages, it has rules. It
differs from natural language in two important ways: 1. The rules are
\emph{simple}. You can learn most of them in a few weeks and gain
reasonable proficiency with the language in a semester. 2. The rules are
\emph{rigid}. If you're proficient in a natural language, you can
understand a non-proficient speaker, glossing over small mistakes. A
computer running R code is not smart enough to do that.

Whenever you write code, you'll make mistakes. When you run a code chunk
that has errors, R will sometimes produce error messages to tell you
what you did wrong.

Errors are okay; even experienced programmers make many errors. When you
make an error, you just have to find the source of the problem, fix it,
and move on.

We have made an error in the next chunk. Run it and see what happens.

\begin{Shaded}
\begin{Highlighting}[]
\FunctionTok{print}\NormalTok{(}\StringTok{"This line is missing something."}
\end{Highlighting}
\end{Shaded}

\begin{verbatim}
## Error in parse(text = input): <text>:2:0: unexpected end of input
## 1: print("This line is missing something."
##    ^
\end{verbatim}

\textbf{Note:} In R Markdown, you can run all chunks by clicking ``Run
All'' in the Run menu. However, the document stops running chunks if it
hits an error, like the one in the chunk above.

You should see an error message. The error tells you that there's a
syntax error - the structure of the code is incorrect. In this case,
we're missing a closing parenthesis.

Try to fix the code above so that you can run the chunk and see the
intended message instead of an error.

\subsection{1.5. The R Environment}\label{the-r-environment}

R maintains an environment that stores all the variables and functions
you've created. You can see what's in your environment in the
Environment pane in RStudio, or by running \texttt{ls()}.

When you run chunks, the results are stored in this environment and can
be used by other chunks. The order in which you run chunks matters!

If you run into problems where your code isn't working as expected, try
restarting R (Session \textgreater{} Restart R) and running all chunks
from the beginning.

\subsection{1.6. Submitting Your Work}\label{submitting-your-work}

All labs in the course will be distributed as R Markdown documents like
this one, and you will submit your work by 1. knitting the document to
HTML and submitting both the .Rmd and .html files. 2. uploading the .Rmd
file to Brightspace.

\section{2. Numbers}\label{numbers}

Quantitative information arises everywhere in data science. In addition
to representing commands to print out lines, expressions can represent
numbers and methods of combining numbers. The expression \texttt{3.2500}
evaluates to the number 3.25. (Run the chunk and see.)

\begin{Shaded}
\begin{Highlighting}[]
\FloatTok{3.2500}
\end{Highlighting}
\end{Shaded}

\begin{verbatim}
## [1] 3.25
\end{verbatim}

Notice that we didn't have to use \texttt{print}. When you run an R
chunk, if the last line has a value, then R helpfully prints out that
value for you. However, it won't print out prior lines automatically.

\begin{Shaded}
\begin{Highlighting}[]
\FunctionTok{print}\NormalTok{(}\DecValTok{2}\NormalTok{)}
\end{Highlighting}
\end{Shaded}

\begin{verbatim}
## [1] 2
\end{verbatim}

\begin{Shaded}
\begin{Highlighting}[]
\DecValTok{3}
\end{Highlighting}
\end{Shaded}

\begin{verbatim}
## [1] 3
\end{verbatim}

\begin{Shaded}
\begin{Highlighting}[]
\DecValTok{4}
\end{Highlighting}
\end{Shaded}

\begin{verbatim}
## [1] 4
\end{verbatim}

Above, you should see that 4 is the value of the last expression, 2 is
printed, but 3 is lost forever because it was neither printed nor last.

You don't want to print everything all the time anyway. But if you feel
sorry for 3, change the chunk above to print it.

\subsection{2.1. Arithmetic}\label{arithmetic}

The line in the next chunk subtracts. Its value is what you'd expect.
Run it.

\begin{Shaded}
\begin{Highlighting}[]
\FloatTok{3.25} \SpecialCharTok{{-}} \FloatTok{1.5}
\end{Highlighting}
\end{Shaded}

\begin{verbatim}
## [1] 1.75
\end{verbatim}

Many basic arithmetic operations are built into R. The common operators
are the same as in most programming languages. The operator that might
be different is \texttt{\^{}}, which raises one number to the power of
the other. So, \texttt{2\^{}3} stands for \(2^3\) and evaluates to 8.

The order of operations is the same as what you learned in elementary
school, and R also has parentheses. For example, compare the outputs of
the chunks below:

\begin{Shaded}
\begin{Highlighting}[]
\DecValTok{6}\SpecialCharTok{+}\DecValTok{6}\SpecialCharTok{*}\DecValTok{5{-}6}\SpecialCharTok{*}\DecValTok{3}\SpecialCharTok{\^{}}\DecValTok{2}\SpecialCharTok{*}\DecValTok{2}\SpecialCharTok{\^{}}\DecValTok{3}\SpecialCharTok{/}\DecValTok{4}\SpecialCharTok{*}\DecValTok{7}
\end{Highlighting}
\end{Shaded}

\begin{verbatim}
## [1] -720
\end{verbatim}

\begin{Shaded}
\begin{Highlighting}[]
\DecValTok{6}\SpecialCharTok{+}\NormalTok{(}\DecValTok{6}\SpecialCharTok{*}\DecValTok{5}\SpecialCharTok{{-}}\NormalTok{(}\DecValTok{6}\SpecialCharTok{*}\DecValTok{3}\NormalTok{))}\SpecialCharTok{\^{}}\DecValTok{2}\SpecialCharTok{*}\NormalTok{((}\DecValTok{2}\SpecialCharTok{\^{}}\DecValTok{3}\NormalTok{)}\SpecialCharTok{/}\DecValTok{4}\SpecialCharTok{*}\DecValTok{7}\NormalTok{)}
\end{Highlighting}
\end{Shaded}

\begin{verbatim}
## [1] 2022
\end{verbatim}

In standard math notation, the first expression is

\[6 + 6 \times 5 - 6 \times 3^2 \times \frac{2^3}{4} \times 7,\]

while the second expression is

\[6 + (6 \times 5 - (6 \times 3))^2 \times \left(\frac{\left(2^3\right)}{4} \times 7\right).\]

\textbf{Question 2.1.1.} Write an R expression in this next chunk that's
equal to
\(\displaystyle 5 \times \left(3 \frac{10}{11}\right) - 50 \frac{1}{3} + 2^{0.5 \times 22} - \frac{7}{33} + 7\).
That's five times three and ten elevenths, minus fifty and a third, plus
two to the power of half twenty-two, minus seven thirty-thirds plus
seven.

Replace the ellipses (\texttt{""}) with your expression. Try to use
parentheses only when necessary.

\textbf{Note:} By ``\(\displaystyle 3 \frac{10}{11}\)'', we mean
\(\displaystyle 3+\frac{10}{11}\), not
\(\displaystyle 3 \times \frac{10}{11}\).

\emph{Hint:} The correct output should start with a familiar number.

\begin{Shaded}
\begin{Highlighting}[]
\CommentTok{\# Replace "" with your expression}
\StringTok{""}
\end{Highlighting}
\end{Shaded}

\begin{verbatim}
## [1] ""
\end{verbatim}

\section{3. Names (Variables)}\label{names-variables}

In natural language, we have terminology that lets us quickly reference
very complicated concepts. We don't say, ``That's a large mammal with
brown fur and sharp teeth!'' Instead, we just say, ``Bear!''

In R, we do this with \emph{assignment statements}. An assignment
statement has a name on the left side of a \texttt{\textless{}-} sign
(or \texttt{=} sign) and an expression to be evaluated on the right.

\begin{Shaded}
\begin{Highlighting}[]
\NormalTok{ten }\OtherTok{\textless{}{-}} \DecValTok{3} \SpecialCharTok{*} \DecValTok{2} \SpecialCharTok{+} \DecValTok{4}
\end{Highlighting}
\end{Shaded}

When you run that chunk, R first computes the value of the expression on
the right-hand side, \texttt{3\ *\ 2\ +\ 4}, which is the number 10.
Then it assigns that value to the name \texttt{ten}. At that point, the
code in the chunk is done running.

After you run that chunk, the value 10 is bound to the name
\texttt{ten}:

\begin{Shaded}
\begin{Highlighting}[]
\NormalTok{ten}
\end{Highlighting}
\end{Shaded}

\begin{verbatim}
## [1] 10
\end{verbatim}

The statement \texttt{ten\ \textless{}-\ 3\ *\ 2\ +\ 4} is not asserting
that \texttt{ten} is already equal to \texttt{3\ *\ 2\ +\ 4}, as we
might expect by analogy with math notation. Rather, that line of code
changes what \texttt{ten} means; it now refers to the value 10, whereas
before it meant nothing at all.

If the designers of R had been ruthlessly pedantic, they might have made
us write

\begin{verbatim}
define the name ten to hereafter have the value of 3 * 2 + 4 
\end{verbatim}

instead. You will probably appreciate the brevity of
``\texttt{\textless{}-}''! But keep in mind that this is the real
meaning.

\textbf{Question 3.1.} Run the following chunk which uses a variable
name \texttt{eleven} that hasn't been assigned to anything. You'll see
an error!

\begin{Shaded}
\begin{Highlighting}[]
\NormalTok{eleven }\SpecialCharTok{+} \DecValTok{8}
\end{Highlighting}
\end{Shaded}

\begin{verbatim}
## Error: object 'eleven' not found
\end{verbatim}

A common pattern in R is to assign a value to a name and then
immediately evaluate the name in the last line in the chunk so that the
value is displayed as output.

\begin{Shaded}
\begin{Highlighting}[]
\NormalTok{close\_to\_pi }\OtherTok{\textless{}{-}} \DecValTok{355}\SpecialCharTok{/}\DecValTok{113}
\NormalTok{close\_to\_pi}
\end{Highlighting}
\end{Shaded}

\begin{verbatim}
## [1] 3.141593
\end{verbatim}

Another common pattern is that a series of lines in a single chunk will
build up a complex computation in stages, naming the intermediate
results.

\begin{Shaded}
\begin{Highlighting}[]
\NormalTok{semimonthly\_salary }\OtherTok{\textless{}{-}} \FloatTok{842.5}
\NormalTok{monthly\_salary }\OtherTok{\textless{}{-}} \DecValTok{2} \SpecialCharTok{*}\NormalTok{ semimonthly\_salary}
\NormalTok{number\_of\_months\_in\_a\_year }\OtherTok{\textless{}{-}} \DecValTok{12}
\NormalTok{yearly\_salary }\OtherTok{\textless{}{-}}\NormalTok{ number\_of\_months\_in\_a\_year }\SpecialCharTok{*}\NormalTok{ monthly\_salary}
\NormalTok{yearly\_salary}
\end{Highlighting}
\end{Shaded}

\begin{verbatim}
## [1] 20220
\end{verbatim}

Names in R can have letters (upper- and lower-case letters are both okay
and count as different letters), underscores, periods, and numbers. The
first character can't be a number (otherwise a name might look like a
number). And names can't contain spaces, since spaces are used to
separate pieces of code from each other.

Other than those rules, what you name something doesn't matter \emph{to
R}. For example, this chunk does the same thing as the above chunk,
except everything has a different name:

\begin{Shaded}
\begin{Highlighting}[]
\NormalTok{a }\OtherTok{\textless{}{-}} \FloatTok{842.5}
\NormalTok{b }\OtherTok{\textless{}{-}} \DecValTok{2} \SpecialCharTok{*}\NormalTok{ a}
\NormalTok{c }\OtherTok{\textless{}{-}} \DecValTok{12}
\NormalTok{d }\OtherTok{\textless{}{-}}\NormalTok{ c }\SpecialCharTok{*}\NormalTok{ b}
\NormalTok{d}
\end{Highlighting}
\end{Shaded}

\begin{verbatim}
## [1] 20220
\end{verbatim}

\textbf{However}, names are very important for making your code
\emph{readable} to yourself and others. The chunk above is shorter, but
it's totally useless without an explanation of what it does.

\subsection{3.1. Checking Your Code}\label{checking-your-code}

Now that you know how to name things, you can start checking whether
your work is correct.

\textbf{Question 3.1.2.} Assign the name \texttt{seconds\_in\_a\_decade}
to the number of seconds between midnight January 1, 2010 and midnight
January 1, 2020. Note that there are two leap years in this span of a
decade. A non-leap year has 365 days and a leap year has 366 days.

\emph{Hint:} If you're stuck, think about how many days are in the
decade, then hours, then minutes, then seconds.

\begin{Shaded}
\begin{Highlighting}[]
\CommentTok{\# Change the next line }
\CommentTok{\# so that it computes the number of seconds in a decade }
\CommentTok{\# and assigns that number the name, seconds\_in\_a\_decade.}

\NormalTok{seconds\_in\_a\_decade }\OtherTok{\textless{}{-}} \StringTok{""}  \CommentTok{\# Replace "" with your calculation}

\CommentTok{\# We\textquotesingle{}ve put this line in this chunk }
\CommentTok{\# so that it will print the value you\textquotesingle{}ve given to seconds\_in\_a\_decade when you run it.  }
\CommentTok{\# You don\textquotesingle{}t need to change this.}
\NormalTok{seconds\_in\_a\_decade}
\end{Highlighting}
\end{Shaded}

\begin{verbatim}
## [1] ""
\end{verbatim}

The correct answer should be 315532800 seconds.

\section{3.2. Comments}\label{comments}

You may have noticed these lines in the chunk where you answered
Question 3.1.2:

\begin{verbatim}
# Change the next line 
# so that it computes the number of seconds in a decade 
# and assigns that number the name, seconds_in_a_decade.
\end{verbatim}

This is called a \emph{comment}. It doesn't make anything happen in R; R
ignores anything on a line after a \texttt{\#}. Instead, it's there to
communicate something about the code to you, the human reader. Comments
are extremely useful.

\subsection{3.3. Application: A Physics
Experiment}\label{application-a-physics-experiment}

On the Apollo 15 mission to the Moon, astronaut David Scott famously
replicated Galileo's physics experiment in which he showed that gravity
accelerates objects of different mass at the same rate. Because there is
no air resistance for a falling object on the surface of the Moon, even
two objects with very different masses and densities should fall at the
same rate. David Scott compared a feather and a hammer.

Here's the transcript of the video:

\textbf{167:22:06 Scott}: Well, in my left hand, I have a feather; in my
right hand, a hammer. And I guess one of the reasons we got here today
was because of a gentleman named Galileo, a long time ago, who made a
rather significant discovery about falling objects in gravity fields.
And we thought where would be a better place to confirm his findings
than on the Moon. And so we thought we'd try it here for you. The
feather happens to be, appropriately, a falcon feather for our Falcon.
And I'll drop the two of them here and, hopefully, they'll hit the
ground at the same time.

\textbf{167:22:43 Scott}: How about that!

\textbf{167:22:45 Allen}: How about that! (Applause in Houston)

\textbf{167:22:46 Scott}: Which proves that Mr.~Galileo was correct in
his findings.

\textbf{Newton's Law.} Using this footage, we can also attempt to
confirm another famous bit of physics: Newton's law of universal
gravitation. Newton's laws predict that any object dropped near the
surface of the Moon should fall

\[\frac{1}{2} G \frac{M}{R^2} t^2 \text{ meters}\]

after \(t\) seconds, where \(G\) is a universal constant, \(M\) is the
moon's mass in kilograms, and \(R\) is the moon's radius in meters. So
if we know \(G\), \(M\), and \(R\), then Newton's laws let us predict
how far an object will fall over any amount of time.

To verify the accuracy of this law, we will calculate the difference
between the predicted distance the hammer drops and the actual distance.
(If they are different, it might be because Newton's laws are wrong, or
because our measurements are imprecise, or because there are other
factors affecting the hammer for which we haven't accounted.)

Someone studied the video and estimated that the hammer was dropped 113
cm from the surface. Counting frames in the video, the hammer falls for
1.2 seconds (36 frames).

\textbf{Question 3.3.1.} Complete the code in the next chunk to fill in
the data from the experiment.

\emph{Hint:} No computation required; just fill in data from the
paragraph above.

\begin{Shaded}
\begin{Highlighting}[]
\CommentTok{\# t, the duration of the fall in the experiment, in seconds.}
\CommentTok{\# Fill this in.}
\NormalTok{time }\OtherTok{\textless{}{-}} \StringTok{""}

\CommentTok{\# The estimated distance the hammer actually fell, in meters.}
\CommentTok{\# Fill this in (remember to convert from cm to meters).}
\NormalTok{estimated\_distance\_m }\OtherTok{\textless{}{-}} \StringTok{""}
\end{Highlighting}
\end{Shaded}

\textbf{Question 3.3.2.} Now, complete the code in the next chunk to
compute the difference between the predicted and estimated distances (in
meters) that the hammer fell in this experiment.

This just means translating the formula above
(\(\frac{1}{2}G\frac{M}{R^2}t^2\)) into R code. You'll have to replace
each variable in the math formula with the name we gave that number in R
code.

\emph{Hint:} Try to use variables you've already defined in question
3.3.1

\begin{Shaded}
\begin{Highlighting}[]
\CommentTok{\# First, we\textquotesingle{}ve written down the values of the 3 universal constants }
\CommentTok{\# that show up in Newton\textquotesingle{}s formula.}

\CommentTok{\# G, the universal constant measuring the strength of gravity.}
\NormalTok{gravity\_constant }\OtherTok{\textless{}{-}} \FloatTok{6.674} \SpecialCharTok{*} \DecValTok{10}\SpecialCharTok{\^{}{-}}\DecValTok{11}

\CommentTok{\# M, the moon\textquotesingle{}s mass, in kilograms.}
\NormalTok{moon\_mass\_kg }\OtherTok{\textless{}{-}} \FloatTok{7.34767309} \SpecialCharTok{*} \DecValTok{10}\SpecialCharTok{\^{}}\DecValTok{22}

\CommentTok{\# R, the radius of the moon, in meters.}
\NormalTok{moon\_radius\_m }\OtherTok{\textless{}{-}} \FloatTok{1.737} \SpecialCharTok{*} \DecValTok{10}\SpecialCharTok{\^{}}\DecValTok{6}

\CommentTok{\# The distance the hammer should have fallen }
\CommentTok{\# over the duration of the fall, in meters, }
\CommentTok{\# according to Newton\textquotesingle{}s law of gravity.  }
\CommentTok{\# The text above describes the formula}
\CommentTok{\# for this distance given by Newton\textquotesingle{}s law.}
\CommentTok{\# **YOU FILL THIS PART IN.**}
\NormalTok{predicted\_distance\_m }\OtherTok{\textless{}{-}} \DecValTok{0}
\NormalTok{estimated\_distance\_m }\OtherTok{\textless{}{-}} \DecValTok{0}

\CommentTok{\# Here we\textquotesingle{}ve computed the difference }
\CommentTok{\# between the predicted fall distance and the distance we actually measured.}
\CommentTok{\# If you\textquotesingle{}ve filled in the above code, this should just work.}
\NormalTok{difference }\OtherTok{\textless{}{-}}\NormalTok{ predicted\_distance\_m }\SpecialCharTok{{-}}\NormalTok{ estimated\_distance\_m}
\NormalTok{difference}
\end{Highlighting}
\end{Shaded}

\begin{verbatim}
## [1] 0
\end{verbatim}

\section{4. Calling Functions}\label{calling-functions}

The most common way to combine or manipulate values in R is by calling
functions. R comes with many built-in functions that perform common
operations.

For example, the \texttt{abs} function takes a single number as its
argument and returns the absolute value of that number. Run the next two
chunks and see if you understand the output.

\begin{Shaded}
\begin{Highlighting}[]
\FunctionTok{abs}\NormalTok{(}\DecValTok{5}\NormalTok{)}
\end{Highlighting}
\end{Shaded}

\begin{verbatim}
## [1] 5
\end{verbatim}

\begin{Shaded}
\begin{Highlighting}[]
\FunctionTok{abs}\NormalTok{(}\SpecialCharTok{{-}}\DecValTok{5}\NormalTok{)}
\end{Highlighting}
\end{Shaded}

\begin{verbatim}
## [1] 5
\end{verbatim}

\subsection{4.1. Application: Computing Walking
Distances}\label{application-computing-walking-distances}

Chunhua is on the corner of 7th Avenue and 42nd Street in Midtown
Manhattan, and she wants to know far she'd have to walk to get to
Gramercy School on the corner of 10th Avenue and 34th Street.

She can't cut across blocks diagonally, since there are buildings in the
way. She has to walk along the sidewalks. Using the map below, she sees
she'd have to walk 3 avenues (long blocks) and 8 streets (short blocks).
In terms of the given numbers, she computed 3 as the difference between
7 and 10, \emph{in absolute value}, and 8 similarly.

Chunhua also knows that blocks in Manhattan are all about 80m by 274m
(avenues are farther apart than streets). So in total, she'd have to
walk \((80 \times |42 - 34| + 274 \times |7 - 10|)\) meters to get to
the park.

\begin{figure}
\centering
\pandocbounded{\includegraphics[keepaspectratio,alt={Map}]{map.jpg}}
\caption{Map}
\end{figure}

\textbf{Question 4.1.1.} Fill in the line
\texttt{num\_avenues\_away\ \textless{}-\ ""} in the next chunk so that
the chunk calculates the distance Chunhua must walk and gives it the
name \texttt{manhattan\_distance}. Everything else has been filled in
for you. \textbf{Use the \texttt{abs} function.}

\begin{Shaded}
\begin{Highlighting}[]
\CommentTok{\# Here\textquotesingle{}s the number of streets away:}
\NormalTok{num\_streets\_away }\OtherTok{\textless{}{-}} \FunctionTok{abs}\NormalTok{(}\DecValTok{42{-}34}\NormalTok{)}

\CommentTok{\# Compute the number of avenues away in a similar way:}
\NormalTok{num\_avenues\_away }\OtherTok{\textless{}{-}} \DecValTok{0}

\NormalTok{street\_length\_m }\OtherTok{\textless{}{-}} \DecValTok{80}
\NormalTok{avenue\_length\_m }\OtherTok{\textless{}{-}} \DecValTok{274}

\CommentTok{\# Now we compute the total distance Chunhua must walk.}
\NormalTok{manhattan\_distance }\OtherTok{\textless{}{-}}\NormalTok{ street\_length\_m}\SpecialCharTok{*}\NormalTok{num\_streets\_away }\SpecialCharTok{+}\NormalTok{ avenue\_length\_m}\SpecialCharTok{*}\NormalTok{num\_avenues\_away}

\CommentTok{\# We\textquotesingle{}ve included this line so that you see the distance you\textquotesingle{}ve computed }
\CommentTok{\# when you run this chunk.  }
\CommentTok{\# You don\textquotesingle{}t need to change it, but you can if you want.}
\NormalTok{manhattan\_distance}
\end{Highlighting}
\end{Shaded}

\begin{verbatim}
## [1] 640
\end{verbatim}

\subparagraph{Multiple arguments}\label{multiple-arguments}

Some functions take multiple arguments, separated by commas. For
example, the built-in \texttt{max} function returns the maximum argument
passed to it.

\begin{Shaded}
\begin{Highlighting}[]
\FunctionTok{max}\NormalTok{(}\DecValTok{2}\NormalTok{, }\SpecialCharTok{{-}}\DecValTok{3}\NormalTok{, }\DecValTok{4}\NormalTok{, }\SpecialCharTok{{-}}\DecValTok{5}\NormalTok{)}
\end{Highlighting}
\end{Shaded}

\begin{verbatim}
## [1] 4
\end{verbatim}

\section{5. Understanding Nested
Expressions}\label{understanding-nested-expressions}

Function calls and arithmetic expressions can themselves contain
expressions. You saw an example in the last question:

\begin{verbatim}
abs(42-34)
\end{verbatim}

has 2 number expressions in a subtraction expression in a function call
expression. And you probably wrote something like \texttt{abs(7-10)} to
compute \texttt{num\_avenues\_away}.

Nested expressions can turn into complicated-looking code. However, the
way in which complicated expressions break down is very regular.

Suppose we are interested in lengths of cats that are very unusual.
We'll say that a length is unusual to the extent that it's far away on
the number line from the average cat length. An estimate of the average
cat length (averaging, we hope, over all cats on Earth today) is
\textbf{18.2} inches.

So if Ravioli is 21.7 inches long, then her length is \(|21.7 - 18.2|\),
or \(3.5\), inches away from the average. Here's a picture of that:

\begin{figure}
\centering
\pandocbounded{\includegraphics[keepaspectratio,alt={Cat Lengths}]{cat_lengths.png}}
\caption{Cat Lengths}
\end{figure}

The source for average cat length is
\href{https://en.wikipedia.org/wiki/Cat\#:~:text=The\%20domestic\%20cat\%20has\%20a,(9\%20and\%2011\%20lb).}{Wikipedia}.
The listed lengths for cats are not real and may not be plausible (but
the names are of real cats!)

And here's how we'd write that expression in one line of R code:

\begin{Shaded}
\begin{Highlighting}[]
\FunctionTok{abs}\NormalTok{(}\FloatTok{21.7} \SpecialCharTok{{-}} \FloatTok{18.2}\NormalTok{)}
\end{Highlighting}
\end{Shaded}

\begin{verbatim}
## [1] 3.5
\end{verbatim}

What's going on here? \texttt{abs} takes just one argument, so the stuff
inside the parentheses is all part of that \emph{single argument}.
Specifically, the argument is the value of the expression
\texttt{21.7\ -\ 18.2}. The value of that expression is \texttt{3.5}.
That value is the argument to \texttt{abs}. The absolute value of that
is \texttt{3.5}, so \texttt{3.5} is the value of the full expression
\texttt{abs(21.7\ -\ 18.2)}.

Picture simplifying the expression in several steps:

\begin{enumerate}
\def\labelenumi{\arabic{enumi}.}
\tightlist
\item
  \texttt{abs(21.7\ -\ 18.2)}
\item
  \texttt{abs(3.5)}
\item
  \texttt{3.5}
\end{enumerate}

In fact, that's basically what R does to compute the value of the
expression.

\textbf{Question 5.1.} Say that Genghis's length is 16.7 inches. In the
next chunk, use \texttt{abs} to compute the absolute value of the
difference between Genghis's length and the average cat length. Give
that value the name \texttt{genghis\_distance\_from\_average\_in}.

\begin{Shaded}
\begin{Highlighting}[]
\CommentTok{\# Replace the "" with an expression }
\CommentTok{\# to compute the absolute value }
\CommentTok{\# of the difference between Genghis\textquotesingle{}s length (16.7 in) and the average cat length.}
\NormalTok{genghis\_distance\_from\_average\_in }\OtherTok{\textless{}{-}} \StringTok{""}

\CommentTok{\# Again, we\textquotesingle{}ve written this here }
\CommentTok{\# so that the distance you compute will get printed }
\CommentTok{\# when you run this chunk.}
\NormalTok{genghis\_distance\_from\_average\_in}
\end{Highlighting}
\end{Shaded}

\begin{verbatim}
## [1] ""
\end{verbatim}

\subsection{5.1. More Nesting}\label{more-nesting}

Now say that we want to compute the more unusual of the two cat lengths.
We'll use the function \texttt{max}, which (again) takes two numbers as
arguments and returns the larger of the two arguments. Combining that
with the \texttt{abs} function, we can compute the larger distance from
average among the two lengths:

\begin{Shaded}
\begin{Highlighting}[]
\CommentTok{\# Just read and run this chunk.}

\NormalTok{ravioli\_length\_in }\OtherTok{\textless{}{-}} \FloatTok{21.7}
\NormalTok{genghis\_length\_in }\OtherTok{\textless{}{-}} \FloatTok{16.7}
\NormalTok{average\_cat\_length }\OtherTok{\textless{}{-}} \FloatTok{18.2}

\CommentTok{\# The larger distance from the average cat length, among the two lengths:}
\NormalTok{larger\_distance\_in }\OtherTok{\textless{}{-}} \FunctionTok{max}\NormalTok{(}\FunctionTok{abs}\NormalTok{(ravioli\_length\_in }\SpecialCharTok{{-}}\NormalTok{ average\_cat\_length), }\FunctionTok{abs}\NormalTok{(genghis\_length\_in }\SpecialCharTok{{-}}\NormalTok{ average\_cat\_length))}

\CommentTok{\# Print out our results in a nice readable format:}
\FunctionTok{print}\NormalTok{(}\FunctionTok{paste}\NormalTok{(}\StringTok{"The larger distance from the average length among these two cats is"}\NormalTok{, larger\_distance\_in, }\StringTok{"inches."}\NormalTok{))}
\end{Highlighting}
\end{Shaded}

\begin{verbatim}
## [1] "The larger distance from the average length among these two cats is 3.5 inches."
\end{verbatim}

The line where \texttt{larger\_distance\_in} is computed looks
complicated, but we can break it down into simpler components just like
we did before.

The basic recipe is to repeatedly simplify small parts of the
expression: * \textbf{Basic expressions:} Start with expressions whose
values we know, like names or numbers. - Examples:
\texttt{genghis\_length\_in} or \texttt{16.7}. * \textbf{Find the next
simplest group of expressions:} Look for basic expressions that are
directly connected to each other. This can be by arithmetic or as
arguments to a function call. - Example:
\texttt{genghis\_length\_in\ -\ average\_cat\_length}. *
\textbf{Evaluate that group:} Evaluate the arithmetic expression or
function call. Use the value computed to replace the group of
expressions.\\
- Example: \texttt{genghis\_length\_in\ -\ average\_cat\_length} becomes
\texttt{-1.3}. * \textbf{Repeat:} Continue this process, using the value
of the previously-evaluated expression as a new basic expression. Stop
when we've evaluated the entire expression. - Example:
\texttt{abs(-1.3)} becomes \texttt{1.3}, and \texttt{max(3.5,\ 1.3)}
becomes \texttt{3.5}.

Ok, your turn.

\textbf{Question 5.1.1.} Given the lengths of Yanay's cats Hummus,
Gatkes, and Zeepty, write an expression that computes the smallest
difference between any of the three lengths. Your expression shouldn't
have any numbers in it, only function calls and the names
\texttt{hummus}, \texttt{gatkes}, and \texttt{zeepty}. Give the value of
your expression the name \texttt{min\_length\_difference}.

\begin{Shaded}
\begin{Highlighting}[]
\CommentTok{\# The three cats\textquotesingle{} lengths, in inches:}
\NormalTok{hummus }\OtherTok{\textless{}{-}} \FloatTok{24.5}  \CommentTok{\# Hummus is 24.5 inches long}
\NormalTok{gatkes }\OtherTok{\textless{}{-}} \FloatTok{19.7}  \CommentTok{\# Gatkes is 19.7 inches long}
\NormalTok{zeepty }\OtherTok{\textless{}{-}} \FloatTok{15.8}  \CommentTok{\# Zeepty is 15.8 inches long}
             
\CommentTok{\# We\textquotesingle{}d like to look at all 3 pairs of lengths, }
\CommentTok{\# compute the absolute difference between each pair, }
\CommentTok{\# and then find the smallest of those 3 absolute differences.  }
\CommentTok{\# This is left to you!  }
\CommentTok{\# If you\textquotesingle{}re stuck, try computing the value for each step of the process }
\CommentTok{\# (like the difference between Hummus\textquotesingle{}s length and Gatkes\textquotesingle{}s length) }
\CommentTok{\# on a separate line and giving it a name (like hummus\_gatkes\_length\_diff)}
\NormalTok{min\_length\_difference }\OtherTok{\textless{}{-}} \StringTok{""}
\end{Highlighting}
\end{Shaded}

You're done with Lab 1!

\textbf{Important submission information:} Be sure to run all chunks and
verify that they work correctly, then knit your document to HTML. Submit
both the .Rmd file and the .html file to Gradescope.

Dash wants to congratulate you and welcome you to Data 8!

\begin{figure}
\centering
\includegraphics[width=3.125in,height=\textheight,keepaspectratio,alt={Dash}]{dash.JPG}
\caption{Dash}
\end{figure}

\end{document}
